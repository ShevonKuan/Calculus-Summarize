%模板
\documentclass[12pt,a4paper]{book}

\usepackage{ctex}
\usepackage{graphicx}%插入图片的宏包
\usepackage{makeidx}
\usepackage{amsmath}
\usepackage{amsfonts}
\usepackage{amssymb}%数学符号宏包
\usepackage{textcomp}%树叶图案在这个包里
\usepackage{bbding}%很多漂亮的图案
\usepackage[dvipsnames, svgnames, x11names]{xcolor}%导入了所有颜色配置文件的宏包
\usepackage{enumerate}%使用改宏包优化罗列环境
\usepackage{CJKfntef}
\usepackage{ulem}%下划线宏包用法和样式如下:
%\uuline{双下划线}
%\uwave{波浪线}
%\sout{中间删除线}
%\xout{斜删除线}
%\dashuline{虚线}
%\dotuline{加点}
\usepackage{geometry}%页边距调整
\geometry{left=1.5cm,right=1.5cm,bottom=1.8cm,top=1.8cm}
\usepackage{titletoc}%目录页的宏包
\usepackage{titlesec}%改变章节或标题的样式的宏包
\usepackage[bookmarks=true,colorlinks,linkcolor=black]{hyperref}

%各类设置
%章节或标题的样式
\titleformat{\section}%设置section的样式
{\raggedright\Large\bfseries}{\includegraphics[width=0.03\textwidth]{sigma.eps}\thesection\hspace*{0.6cm}}{0pt}{}
\titlespacing*{\subsection}{3em}{3em}{2em}[1em]
\titleformat{\subsection}%设置section的样式
{\raggedright\large\bfseries}{\hspace*{1cm}\thesubsection\hspace*{0.6cm}}{0pt}{}
\titlespacing*{\subsection}{3em}{2em}{2em}[1em]
%格式如下:\titlespacing*{章节名称}{左间距}{(前)行间距}{(后)行间距}[右间距(一般都没用,填0.1em即可,但不能不填)]
\titlespacing*{\subsubsection}{2em}{1em}{1em}[1em]

%制作索引
\makeindex

%公式编号设置
\numberwithin{equation}{section}

%字体设置
\setCJKmainfont[BoldFont=STZHONGS.ttf]{STZHONGS.ttf}%需要查看电脑字体查找对应字体的文件英文文件名
\setCJKfamilyfont{song}{STZHONGS.ttf}
\setCJKfamilyfont{kai}{simkai.ttf}%都是用来定义字体的(此处使用电脑自带楷书)

%定义区域
\definecolor{zs}{HTML}{2a7ae2}%定义某个颜色,对应颜色代号查表
\definecolor{dy}{HTML}{FF359A}
\definecolor{dl}{HTML}{4B0091}
\definecolor{ff}{HTML}{007500}
\definecolor{bt}{HTML}{5B00AE}

		%定义环境
\newtheorem{method}{\hspace*{0.3cm}\color{ff}\textleaf 方法}[section]
\newtheorem{defination}{\hspace*{0.3cm}\color{dy}\FiveFlowerOpen \hspace*{0.2cm}定义}[section]
\newtheorem{feature}{\color{1a9850}$ \square  $性质}[section]
\newtheorem{inference}{\color{00ba38}$ \square  $推论}[section]
\newtheorem{theorem}{\hspace*{0.3cm}\color{dl} $ \square  $定理}[section]
\newtheorem{example}{\color{53a9ab}$ \square  $例}[section]
\newtheorem{proof}{证明}[chapter]

%文章标题
		\title{微分方程总结}\author{易鹏}

%调整间距(倍数)
\linespread{1.5}

%标题及目录
\begin{document}
\maketitle %显示标题
\newpage
\pagenumbering{Roman}
\setcounter{page}{0}%强行设置起始页码
\tableofcontents
\thispagestyle{empty}


%正文部分
\newpage
\pagenumbering{arabic}
\setcounter{page}{1}
\chapter{微分方程}
\section{微分方程的基本概念}
\vspace*{0.3cm}

\begin{defination}
    \hspace*{0.3cm}一般地,凡表示未知函数、未知函数的导数与自变量之间的关系的方程,叫做\uuline{微分方程}\index{微分方程!微分方程},有时也简称\uuline{方程}.
\end{defination}

\begin{defination}
    \hspace*{0.3cm}微分方程中所出现的未知函数的最高阶导数的阶数,叫做\uuline{微分方程的阶}\index{微分方程!微分方程的阶}.例如三阶微分方程$x^3y'''+x^2y''-4xy'=3x^2+\sin{2x}$.一般地,$n$阶微分方程的形式是
    \begin{equation}
        F(x,y,y',\cdots ,y^{(n)})=0\label{n阶微分方程}
    \end{equation}
    其中$y^{(n)}$是必须有的,而其余项$x,y,y',\cdots ,y^{(n-1)}$可有可无.
\end{defination}

\begin{defination}
    \hspace*{0.3cm}找出这样的函数,把这个函数代入微分方程\eqref{n阶微分方程}能使该方程成为恒等式.这个函数就叫做该\uuline{微分方程的解}\index{微分方程!微分方程的解}.设函数$y=\varphi(x)$在区间$I$上有$n$阶连续导数,如果在区间$I$上,
    $$F\left[x,\varphi(x),\varphi'(x),\cdots ,\varphi^{(n)}(x)\right]\equiv 0$$
    那么函数就叫做微分方程\eqref{n阶微分方程}在区间$I$上的解.
\end{defination}

\begin{defination}
    \hspace*{0.3cm}如果微分方程的解中含有任意常数,且任意常数的个数与微分方程的阶数相同\footnote{这里所说的任意常数是相互独立的, 就是说,它们不能合并而使得任意常数的个数减少(参见本章第六节关于函数组的线性相关性)},这样的解叫做\uuline{微分方程的通解}\index{微分方程!微分方程的通解}.
\end{defination}

\hspace*{0.6cm}通解中含有任意常数,所以它还不能完全确定地反映某一客观事物的规律性.所以为了完全确定地反映客观事物的规律性,必须确定这些常数的值.为此要根据问题的实际情况,提出确定这些常数的条件.例如设一阶微分方程中的未知函数为$y=\varphi (x)$,通常给出的条件为$x=x_0,y=y_0$,也记为${y|}_{x=x_0}=y_0$.

\begin{defination}
    \hspace*{0.3cm}上述给出的条件就称为\uuline{初值条件}\index{微分方程!初值条件}.确定了通解中的任意常数以后,得到的解就叫做\uuline{微分方程的特解}\index{微分方程!微分方程的特解}.求微分方程$y'=f(x ,y)$满足初值条件${y|}_{x=x_0}=y_0$的特解这样一个问题,叫做一阶微分方程的\uuline{初值问题}\index{微分方程!初值问题},记作
    \begin{equation}
        \left\lbrace
        \begin{array}{l}
            y' = f(x,y) \\
            y\left | {_{x = {x_0}} = {y_0}} \right.
        \end{array}
        \right.
        \label{初值问题1}
    \end{equation}
\end{defination}

\begin{defination}
    微分方程的解的图形是一条曲线,叫做微分方程的\uuline{积分曲线}\index{微分方程!积分曲线}.初值问题\eqref{初值问题1}的几何意义,就是求微分方程\eqref{初值问题1}的通过点$(x_0,y_0)$的那条积分曲线,二阶微分方程的初值问题
    \begin{equation}
        \left\lbrace
        \begin{array}{l}
            y''= f(x,y,y')                          \\
            y\left | {_{x = {x_0}} = {y_0}} \right. \\
            y'\left | {_{x = {x_0}} = {y_0'}} \right.
        \end{array}
        \right.
        \label{初值问题2}
    \end{equation}
    的几何意义是求微分方程\eqref{初值问题2}的通过点$(x_0,y_0)$的且在该点处的切线斜率为$y_0'$的那条积分曲线.
\end{defination}

\section{各种微分方程的求解}
\subsection{可分离变量的微分方程}

\begin{defination}
    \hspace*{0.3cm}一般地,如果一个微分方程能写成
    \begin{equation}
        g(y)\,\mathrm{d}y=f(x)\,\mathrm{d}x
        \label{可分离变量的微分方程}
    \end{equation}
    的形式,就是说,能把微分方程写成一端只含$y$的函数和$\mathrm{d}y$,另一端写成只含$x$的函数和$\mathrm{d}x$的形式,那么原方程就称为\uuline{可分离变量的微分方程}\index{微分方程!可分离变量的微分方程}.
\end{defination}

\begin{method}
    \hspace*{0.3cm}方程\eqref{可分离变量的微分方程}的解法,只需要两边同时积分即可,即
    \begin{equation}
        \int g(y)\,\mathrm{d}y=\int f(x)\,\mathrm{d}x
        \label{可分离变量的微分方程2}
    \end{equation}
    设$G(y)$和$F(x)$分别为$g(y)$和$f(x)$的原函数,那么
    \begin{equation}
        G(y)=F(x)+C
        \label{可分离变量的微分方程3}
    \end{equation}
\end{method}

\begin{defination}
    \hspace*{0.3cm}如果方程\eqref{可分离变量的微分方程3}表示的是隐函数$y=\varphi (x)$,那么式\eqref{可分离变量的微分方程3}是微分方程\eqref{可分离变量的微分方程}的\uuline{隐式解}\index{微分方程!隐式解}.又由于关系式\eqref{可分离变量的微分方程3}中含有任意常数,因此式\eqref{可分离变量的微分方程3}微分方程\eqref{可分离变量的微分方程}的通解,也叫做\uuline{隐式通解}\index{微分方程!隐式通解}.
\end{defination}
\vspace{0.3cm}

\subsection{齐次方程}

\begin{defination}
    \hspace*{0.3cm}如果一阶微分方程可以化成

    \begin{equation}
        \frac{\mathrm{d}y}{\mathrm{d}x}=\varphi (\frac{y}{x})
        \label{齐次方程1}
    \end{equation}

    的形式,那么这个微分方程就叫做\uuline{齐次方程}\index{微分方程!齐次方程}.
\end{defination}

\begin{method}
    \hspace*{0.3cm}求解齐次微分方程\eqref{齐次方程1}的方法如下:\\
    第一步:换元
    $$u=\frac{y}{x}$$
    其中$u$是关于$x$的一个新函数,所以
    $$y=ux,\frac{\mathrm{d}y}{\mathrm{d}x}=u+\frac{\mathrm{d}u}{\mathrm{d}x}x$$
    第二步:代入式\eqref{齐次方程1},得
    $$u+\frac{\mathrm{d}u}{\mathrm{d}x}x=\varphi (u)$$
    第三步:分离变量,得到
    $$\frac{1}{\varphi (u)-u}\,\mathrm{d}u=\frac{1}{x}\,\mathrm{d}x$$
    第四步:两边同时积分,得到
    $$\int \frac{1}{\varphi (u)-u}\,\mathrm{d}u=\ln|x|\,$$
    即
    $$x=\mathrm{e}^{\int \frac{1}{\varphi (u)-u}\,\mathrm{d}u}$$
\end{method}

\begin{theorem}
    \hspace{0.3cm}齐次方程\eqref{齐次方程1}的通解为
    \begin{equation}
        x=\mathrm{e}^{\int \frac{1}{\varphi (u)-u}\,\mathrm{d}u}
        \label{齐次方程的通解}
    \end{equation}

\end{theorem}


\subsection{可化为齐次的方程}
\begin{method}
    \hspace{0.3cm}方程
    \begin{equation}
        \frac{\mathrm{d}y}{\mathrm{d}x}=\frac{a_1x+b_1y+c_1}{a_2x+b_2y+c_2}
        \label{非齐次方程}
    \end{equation}
    当$c_1=c_2=0$时方程\eqref{非齐次方程}是齐次的,否则不是齐次的.在非齐次的情形,可以用变换法和待定系数法使得其变为齐次方程.令
    $$x=X+h,y=Y+k,$$
    于是,
    $$\mathrm{d}x=\mathrm{d}X,\mathrm{d}y=\mathrm{d}Y,$$
    带入方程\eqref{非齐次方程}得到
    $$\frac{\mathrm{d}Y}{\mathrm{d}X}=\frac{a_1X+b_1Y+a_1h+b_1k+c_1}{a_2X+b_2Y+a_2h+b_2k+c_2}.$$
    那么要使方程\eqref{非齐次方程}是齐次方程,那么需要满足方程组
    $$\left\lbrace
        \begin{array}{l}
            a_1h+b_1k+c_1=0 \\
            a_2h+b_2k+c_2=0
        \end{array}
        \right.$$
    如果上述方程组的系数行列式
    $\left| \begin{array}{cc}
            a_1 & b_1 \\
            a_2 & b_2
        \end{array}  \right| \ne 0$
    ,即$\displaystyle\frac{a_2}{a_1}\ne \frac{b_1}{b_2} $,那么这个方程组存在唯一的$h$和$k$使得上述方程组成立.那么,方程\eqref{非齐次方程}便化为齐次方程
    $$\frac{\mathrm{d}Y}{\mathrm{d}X}=\frac{a_1X+b_1Y}{a_2X+b_2Y}.$$
    求出这个齐次方程的解以后在通解中要记得将元换回来,即代入$X=x-h,Y=y-k$.\vspace*{0.2cm}\\

    \hspace*{0.5cm}特别地,当$\displaystyle\frac{a_2}{a_1}= \frac{b_1}{b_2}$时,无法求出$h$和$k$.这时令$\displaystyle\frac{a_2}{a_1}= \frac{b_1}{b_2}=\lambda $,那么方程\eqref{非齐次方程}可以写为
    $$\frac{\mathrm{d}y}{\mathrm{d}x}=\frac{a_1x+b_1y+c_1}{\lambda (a_1x+b_1y)+c_2}$$
    引入新变量$v=a_1x+b_1y$,那么
    $$\frac{\mathrm{d}v}{\mathrm{d}x}=a_1+b_1 \, \frac{\mathrm{d}y}{\mathrm{d}x} \qquad \frac{\mathrm{d}y}{\mathrm{d}x}=\frac{1}{b_1}\left(\frac{\mathrm{d}v}{\mathrm{d}x}-a_1 \right) $$
    代入方程\eqref{非齐次方程}得到
    $$\frac{1}{b_1}\left(\frac{\mathrm{d}v}{\mathrm{d}x}-a_1 \right)=\frac{v+c_1}{\lambda v+c_2}$$
    这就变成了一个可以分离变量的方程,很容易就可以求解.
\end{method}

\subsection{一阶线性微分方程}




%打印索引—————————————
\newpage
\addcontentsline{toc}{chapter}{附录}
\addcontentsline{toc}{section}{索引}
\appendix
\kaishu
\printindex
%———————————————

\end{document}