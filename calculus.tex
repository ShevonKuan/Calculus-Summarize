\documentclass[10pt,a4paper]{article}
\usepackage{shevonNotebook}
\begin{document}
\numberwithin{equation}{subsection}
\allowdisplaybreaks%强制自动换行
\maketitle
\thispagestyle{empty}
\newpage
\tableofcontents
\thispagestyle{empty}
\newpage
\setcounter{page}{1}
\section{常用三角函数公式}
\subsection{积化和差公式}
\begin{align}
    \sin \alpha \cos \beta & =\frac{1}{2}[\sin (\alpha+\beta)+\sin(\alpha-\beta)]  \\[7pt]
    \cos \alpha \sin \beta & =\frac{1}{2}[\sin (\alpha+\beta)-\sin(\alpha-\beta)]  \\[7pt]
    \cos \alpha \cos \beta & =\frac{1}{2}[\cos (\alpha+\beta)+\cos(\alpha-\beta)]  \\[7pt]
    \sin \alpha \sin \beta & =-\frac{1}{2}[\cos (\alpha+\beta)-\cos(\alpha-\beta)]
\end{align}
\subsection{和差化积公式}
\begin{align}
    \sin\alpha+\sin\beta & =2\sin\frac{\alpha+\beta}{2}\cos\frac{\alpha-\beta}{2}  \\[7pt]
    \sin\alpha-\sin\beta & =2\cos\frac{\alpha+\beta}{2}\sin\frac{\alpha-\beta}{2}  \\[7pt]
    \cos\alpha+\cos\beta & =2\cos\frac{\alpha+\beta}{2}\cos\frac{\alpha-\beta}{2}  \\[7pt]
    \cos\alpha-\cos\beta & =-2\sin\frac{\alpha+\beta}{2}\sin\frac{\alpha-\beta}{2} \\[7pt]
    \tan\alpha+\tan\beta & =\frac{\sin (\alpha+\beta)}{\cos\alpha\cdot\cos \beta}
\end{align}
\subsection{归一化公式}
\begin{align}\label{gyhgs}
    \sin^2 x+\cos^2x  & =1 \\[7pt]
    \sec^2 x-\tan^2x  & =1 \\[7pt]
    \cosh^2x-\sinh^2x & =1
\end{align}
\subsection{倍(半)角公式 \quad 降(升)幂公式}
\begin{align}
    \sin^2x & =\frac{1}{2}(1-\cos 2x)                                                        \\[7pt]
    \cos^2x & =\frac{1}{2}(1+\cos 2x)                                                        \\[7pt]
    \tan^2x & =\frac{1-\cos 2x}{1+\cos 2x}                                                   \\[7pt]
    \sin x  & =2\sin\frac{x}{2}\cos\frac{x}{2}                                               \\[7pt]
    \cos x  & =2\cos^2\frac{x}{2}-1=1-2\sin^2\frac{x}{2}=\cos^2\frac{x}{2}-\sin^2\frac{x}{2} \\[7pt]
    \tan x  & =\frac{2\tan(x/2)}{1-\tan^2(x/2)}
\end{align}
\subsection{万能公式}
令$ u=\tan\dfrac{x}{2} $则
\begin{align}
    \sin x=\frac{2u}{1+u^2} \\[7pt]
    \cos x=\frac{1-u^2}{1+u^2}
\end{align}
\section{常用的佩亚诺型余项泰勒公式}
\mybox[53a9ab]{\leftline{\textbf{泰勒公式\small{ $ (\xi\ \mbox{在}x_0\ \mbox{与}x\ \mbox{之间}) $}:}}\vspace{-2em}
    \begin{align}
        f(x)= & f(x_0)+f'(x_0)(x-x_0)+\frac{f''(x_0)}{2!}(x-x_0)^2+\cdots+\frac{f^{(n)}(x_0)}{n!}(x-x_0)^n+o[(x-x_0)^n]\notag                               \\[7pt]
        f(x)= & f(x_0)+f'(x_0)(x-x_0)+\frac{f''(x_0)}{2!}(x-x_0)^2+\cdots+\frac{f^{(n)}(x_0)}{n!}(x-x_0)^n+\frac{f^{(n+1)}(\xi)}{(n+1)!}(x-x_0)^{n+1}\notag
    \end{align}}

\leftline\small{由此可得常用的泰勒公式}
\begin{align}
    \mathrm{e}^{x} & =1+x+\frac{1}{2}x^{2}+\frac{1}{6}x^{3}+\cdots+\frac{1}{n!}x^{n}+o(x^{n})    \\[7pt]
    \ln(x+1)       & =x-\frac{1}{2}x^2+\frac{1}{3}x^3-\cdots+(-1)^{n-1}\frac{1}{n}x^{n}+o(x^{n})
\end{align}
令 $n=2m$ 有,
\begin{align}
    \sin x & =x-\frac{1}{6}x^{3}+\frac{1}{120}x^{5}+\cdots+(-1)^{m-1}\frac{1}{(2m-1)!}x^{2m-1}+o(x^{2m}) \\[7pt]
    \cos x & =1-\frac{1}{2}x^2+\frac{1}{24}x^4-\cdots+(-1)^m \frac{1}{(2m)!}x^{2m}+o(x^{2m+1})           \\[7pt]
    \tan x & =x+\frac{1}{3}x^3+\frac{2}{15}x^5+\frac{17}{315}x^7+\cdots+o(x^{2m-1})
\end{align}
\small\mbox{此处考虑到tan的泰勒公式其通项公式会出现伯努利数故此处略去其通项 }
\begin{align}
    \arcsin x & =x+\frac{1}{6}x^3+\frac{3}{40}x^{5}+\cdots+o(x^{2m})
\end{align}
常用于近似计算的泰勒公式
\begin{align}
    \frac{1}{1-x}  & =1+x+x^2+x^3+\cdots+x^n+o(x^n)                                                    \\[7pt]
    (1+x)^{\alpha} & =\sum_{i=0}^{n}\frac{\prod_{j=0}^{i-1}{(\alpha-j})}{i!}x^n+o(x^n)\notag           \\[7pt]
                   & =1+\alpha x+\frac{\alpha(\alpha-1)}{2}x^2+\cdots+o(x^n)                           \\[7pt]
    \alpha^x       & =\sum_{i=0}^{n}\frac{\ln^n \alpha}{n!}x^n+o(x^n)\notag                            \\[7pt]
                   & =1+x\ln \alpha+\frac{\ln^2 \alpha}{2}x^2+\cdots+\frac{\ln^n \alpha}{n!}x^n+o(x^n)
\end{align}
\section{基本求导公式}
\begin{multicols}{2}
    \begin{equation}
        \left( C\right)'=0
    \end{equation}
    \begin{equation}
        \left( x^{\mu}\right)'=\mu x^{\mu-1}
    \end{equation}
    \begin{equation}
        \left( \sin x\right)'=\cos x
    \end{equation}
    \begin{equation}
        \left( \cos x\right)'=-\sin x
    \end{equation}
    \begin{equation}
        \left( \tan x\right)'=\sec^2 x
    \end{equation}
    \begin{equation}
        \left( \cot x\right)'=-\csc^2 x
    \end{equation}
    \begin{equation}
        \left( \sec x\right)'=\sec x\cdot\tan x
    \end{equation}
    \begin{equation}
        \left( \csc x\right)'=-\csc x\cdot\tan x
    \end{equation}
    \begin{equation}
        \left( a^x\right)'=a^x\ln a\ (a>0,a\neq1)
    \end{equation}
    \begin{equation}
        \left( \log_{a}x\right)'=\frac{1}{x\cdot\ln a}\ (a>0,a\neq1)
    \end{equation}
    \begin{equation}
        \left( \arcsin x\right)'=\frac{1}{\sqrt{1-x^2}}
    \end{equation}
    \begin{equation}
        \left( \arccos x\right)'=-\frac{1}{\sqrt{1-x^2}}
    \end{equation}
    \begin{equation}
        \left( \arctan x\right)'=\frac{1}{1+x^2}
    \end{equation}
    \begin{equation}
        \left( \mathrm{arccot}\, x\right)'=-\frac{1}{1+x^2}
    \end{equation}
\end{multicols}
\section{函数图形描述中涉及到的重要公式}
\subsection{常用曲率计算公式}
\mybox[53a9ab]{
    曲率的定义式$ K=\displaystyle\left|\frac{\mathrm{d}\alpha}{\mathrm{d}s}\right| $}
由定义式我们可以推得
\begin{enumerate}
    \item \textbf{直角坐标}系中的曲线\ $y=y(x)$\ 有曲率表达式
          \begin{equation}
              K=\frac{\left|y''\right|}{\left( 1+y^{'2} \right)^{3/2}}\mbox{;}
          \end{equation}
    \item \textbf{参数方程}表示的曲线\ $x=\varphi(t),y=\psi(t) $\ 有曲率表达式
          \begin{equation}
              K=\frac{\left|\varphi'(t)\psi''(t)-\varphi''(t)\psi'(t)\right|}{\left[ \varphi^{'2}(t) +\psi^{'2}(t) \right]^{3/2}}\mbox{;}
          \end{equation}
    \item \textbf{极坐标}表示的的曲线\ $y=y(x)$\ 有曲率表达式
          \begin{equation}
              K=\frac{\left|r^2+2r^{'2}-r\cdot r''\right|}{\left(r^2+r^{'2}\right)^{3/2}}\mbox{;}
          \end{equation}
    \item 曲线在对应点\ $M(x,y)$\ 的曲率中心\ $D(\alpha,\beta)$\ 的坐标为
          \begin{equation}
              \begin{cases}
                  \alpha=x-\displaystyle\frac{y'(1+y^{'2})^3}{y^{''2}} \\[7pt]
                  \beta=y+\displaystyle\frac{1+y^{'2}}{y''}
              \end{cases}
          \end{equation}
          .
\end{enumerate}
\subsection{曲线的渐近线}
\begin{enumerate}
    \item 若\ $\lim\limits_{ x\rightarrow \infty }f(x)=b$\ ,则称\ $y=b$\ 为曲线\ $f(x)$\ 的\textbf{水平渐近线}
    \item 若\ $\lim\limits_{ x\rightarrow x_0 }f(x)=\infty$\ ,则称\ $x=x_0$\ 为曲线\ $f(x)$\ 的\textbf{垂直渐近线}
    \item 若\ $\lim\limits_{ x\rightarrow \infty }[f(x)-(ax+b)]=0$\ ,其中
          $
              \begin{cases}
                  a=\displaystyle \lim\limits_{x\to \infty}\frac{f(x)}{x} \\[7pt]
                  b=\displaystyle \lim\limits_{x\to \infty}[f(x)-ax]
              \end{cases}
          $
          则称\ $y=ax+b$\ 为曲线\ $f(x)$\ 的\textbf{斜渐近线}
\end{enumerate}
\section{基本积分公式}
\begin{align}
     & \int k \,\mathrm{d}x=kx+C   \ \mbox{(其中}k\mbox{为常数)}                                                              \\[7pt]
     & \int x^\mu\,\mathrm{d}x=\frac{x^{\mu+1}}{\mu+1}+C\ (\mu\neq-1)                                                         \\[7pt]
     & \int \frac{1}{x}\,\mathrm{d}x=\ln|x|+C                                                                                 \\[7pt]
     & \int\frac{\mathrm{d}x}{1+x^2}=\arctan x+C                                                                              \\[7pt]
     & \int\frac{\mathrm{d}x}{\sqrt{1-x^2}}=\arcsin x+C_1=-\arccos x+C_2                                                      \\[7pt]
     & \int \sin x\,\mathrm{d}x=-\cos x+C                                                                                     \\[7pt]
     & \int\cos x \,\mathrm{d}x=\sin x +C                                                                                     \\[7pt]
     & \int\tan x\,\mathrm{d}x=-\ln |\cos x|+C                                                                                \\[7pt]
     & \int\cot x\,\mathrm{d}x=\ln |\sin x|+C                                                                                 \\[7pt]
     & \int\csc x\,\mathrm{d}x=\int\frac{1}{\sin{x}}\,\mathrm{d}x=\frac{1}{2}
    \ln{\left|\frac{1-\cos{x}}{1+\cos{x}}\right|}+C=\ln{\left|\tan{\frac{x}{2}}\right|}+C=\ln{\left|\csc{x}-\cot{x}\right|}+C \\[7pt]
     & \int\sec x\,\mathrm{d}x=\int\frac{1}{\cos{x}}\,\mathrm{d}x=\frac{1}{2}
    \ln{\left|\frac{1+\sin{x}}{1-\sin{x}}\right|}+C=\ln{\left|\sec{x}+\tan{x}\right|}+C                                       \\[7pt]
     & \int\sec^2 x\,\mathrm{d}x=\tan x +C                                                                                    \\[7pt]
     & \int \csc^2 x\,\mathrm{d}x=-\cot x +C                                                                                  \\[7pt]
     & \int \sec x\cdot\tan x \,\mathrm{d}x=\sec x+C                                                                          \\[7pt]
     & \int\csc x \cdot\cot x \,\mathrm{d}x=-\csc x+C                                                                         \\[7pt]
     & \int \mathrm{e}^x \,\mathrm{d}x=\mathrm{e}^x+C                                                                         \\[7pt]
     & \int a^x\,\mathrm{d}x=\frac{a^x}{\ln a}+C                                                                              \\[7pt]
     & \int \sinh x\,\mathrm{d}x=\cosh x+C                                                                                    \\[7pt]
     & \int \cosh x\,\mathrm{d}x=\sinh x+C                                                                                    \\[7pt]
     & \int \frac{1}{a^2+x^2}\,\mathrm{d}x=\frac{1}{a}\arctan\frac{x}{a}+C                                                    \\[7pt]
     & \int \frac{1}{a^2-x^2}\,\mathrm{d}x=\frac{1}{2a}\ln \left|\frac{a+x}{a-x}\right|+C                                     \\[7pt]
     & \int \frac{1}{\sqrt{a^2-x^2}}\,\mathrm{d}x=\arcsin\frac{x}{a}+C                                                        \\[7pt]
     & \int \frac{1}{\sqrt{x^2\pm a^2}}\,\mathrm{d}x=\ln \left|x+\sqrt{x^2\pm a^2}\right|+C
\end{align}
\section{基本积分方法}
\subsection{第一类换元法}
\subsubsection{三角函数之积的积分}
\begin{enumerate}
    \item 一般地,对于$ \sin^{2k+1}x\cos^n x $ 或$ \sin^n x \cos^{2k+1}x $ (其中$ k\in\mathbb{N} $)型函数的积分,总可依次作变换 $ u=\cos x $或$ u=\sin x $ ,从而求得结果;
    \item 一般地,对于$ \sin^{2k}x\cos^{2l}x $ 或 (其中$ k,l\in \mathbb{N} $)型函数的积分,总是利用降幂公式$ \sin^2=\dfrac{1}{2}(1-\cos 2x),
              \cos^2=\dfrac{1}{2}(1+\cos 2x) $化成$ \cos 2x $的多项式 ,从而求得结果;
    \item 一般地,对于$ \tan^{n}x\sec^{2k} x $ 或$ \tan^{2k-1} x \sec^{n}x $ (其中$ n,k\in\mathbb{N}_{+} $)型函数的积分,总可依次作变换 $ u=\tan x $或$ u=\sec x $ ,从而求得结果;
\end{enumerate}
\subsubsection{常见的凑微分类型}
\begin{align}
     & \int {f( ax + b){\rm{d}}x = }\frac{1}{a}\int {f(ax+b){\mathrm{d}}(ax + b)\;(a \neq 0)}                                                              \\[7pt]
     & \int {f(a{x^{m + 1}} + b){x^m}{\rm{d}}x} = \frac{1}{{a(m + 1)}}\int {f(a{x^{m + 1}} + b){\rm{d}}(a{x^{m + 1}} + b)}                                 \\[7pt]
     & \int {f\left( \frac{1}{x}\right) \frac{{{\rm{d}}x}}{{{x^2}}}\;}  =  - \int {f\left( \frac{1}{x}\right) {\rm{d}}\left( \frac{{\rm{1}}}{x}\right) \;} \\[7pt]
     & \int {f(\ln x)\frac{1}{x}} {\rm{d}}x = \int {f(\ln x){\rm{d(}}\ln x)}                                                                               \\[7pt]
     & \int {f({\mathrm{e}^x})} {\mathrm{e}^x}{\rm{d}}x = \int {f({\mathrm{e}^x}} ){\rm{d(}}{\mathrm{e}^x})                                                \\[7pt]
     & \int {f(\sqrt x } )\frac{{{\rm{d}}x}}{{\sqrt x }} = 2\int {f(\sqrt x } ){\rm{d}}(\sqrt x )                                                          \\[7pt]
     & \int {f(\sin x)\cos x{\rm{d}}x = } \int {f(\sin x){\rm{d}}\sin x}                                                                                   \\[7pt]
     & \int {f(\cos x)\sin x{\rm{d}}x = }  - \int {f(\cos x){\rm{d}}\cos x}                                                                                \\[7pt]
     & \int {f(\tan x){{\sec }^2}} x{\rm{d}}x = \int {f(\tan x){\rm{d}}\tan x}                                                                             \\[7pt]
     & \int {f(\cot x){{\csc }^2}} x{\rm{d}}x =  - \int {f(\cot x){\rm{d}}\cot x}                                                                          \\[7pt]
     & \int {f(\arcsin x)\frac{1}{{\sqrt {1 - {x^2}} }}} {\rm{d}}x = \int {f(\arcsin x){\rm{d}}\arcsin x}                                                  \\[7pt]
     & \int {f(\arctan x)\frac{1}{{1 + {x^2}}}} {\rm{d}}x = \int {f(\arctan x){\rm{d}}\arctan x}                                                           \\[7pt]
     & \int {\frac{{f'(x)}}{{f(x)}}} {\rm{d}}x = \int {\frac{{{\rm{d}}f(x)}}{{f(x)}}}  = \ln \left| f(x)\right| + C
\end{align}
\subsection{有理函数的积分}
\subsubsection{部分分式}
\begin{align}
    \frac{{P(x)}}{{Q(x)}} = & \frac{{{A_1}}}{{{{(x - a)}^\alpha }}} + \frac{{{A_2}}}{{{{(x - a)}^{\alpha  - 1}}}} +  \cdots  + \frac{{{A_\alpha }}}{{x - a}} + \notag                                                                 \\
                            & \frac{{{B_1}}}{{{{(x - b)}^\beta }}} + \frac{{{B_2}}}{{{{(x - b)}^{\beta  - 1}}}} +  \cdots  + \frac{{{B_\beta }}}{{x - b}} + \notag                                                                    \\
                            & \frac{{{M_1}x + {N_1}}}{{{{({x^2} + px + q)}^\lambda }}} + \frac{{{M_2}x + {N_2}}}{{{{({x^2} + px + q)}^{\lambda  - 1}}}} +  \cdots  + \frac{{{M_\lambda }x + {N_\lambda }}}{{{x^2} + px + q}} + \notag \\
                            & \cdots
\end{align}
\subsubsection{三角函数的特殊定积分}
\begin{align}
    I_n & =\int_0^{\frac{\pi}{2}}\sin^nx\,\mathrm{d}x=\int_0^{\frac{\pi}{2}}\cos^nx\,\mathrm{d}x\notag \\
    I_n & =\frac{n-1}{n}I_{n-2}\notag                                                                  \\
        & =\begin{cases}
        \ \dfrac{{n - 1}}{n} \cdot \dfrac{{n - 3}}{{n - 2}} \cdots \dfrac{4}{5} \cdot \dfrac{2}{3}\quad (n\mbox{为大于}1\mbox{的正奇数}),I_1=1 \\[13pt]
        \ \dfrac{{n - 1}}{n} \cdot \dfrac{{n - 3}}{{n - 2}} \cdots \dfrac{3}{4} \cdot \dfrac{1}{2} \cdot \dfrac{\pi }{2}\quad (n\mbox{为正偶数}),I_0=\dfrac{\pi}{2}
    \end{cases}
\end{align}
\section{多元函数微分}
\subsection{偏导数}
\subsubsection{偏导数记法}
设函数$ z=f(x,y) $在区域$ D $内有偏导数:\[\frac{{\partial z}}{{\partial x}} = {f_x}(x,y),\quad \frac{{\partial z}}{{\partial y}} = {f_y}(x,y)\]他们的偏导数若存在,那么称其偏导数为$ z=f(x,y) $的\md{二阶偏导数}.按照对变量求导次序不同,有如下四个二阶偏导数:\begin{align*}
     & \frac{\partial }{{\partial x}}\left( {\frac{{\partial z}}{{\partial x}}} \right) = \frac{{{\partial ^2}z}}{{\partial {x^2}}} = {f_{xx}}(x,y)       & \frac{\partial }{{\partial y}}\left( {\frac{{\partial z}}{{\partial x}}} \right) = \frac{{{\partial ^2}z}}{{\partial x\partial y}} = {f_{xy}}(x,y) \\[7pt]
     & \frac{\partial }{{\partial x}}\left( {\frac{{\partial z}}{{\partial y}}} \right) = \frac{{{\partial ^2}z}}{{\partial y\partial x}} = {f_{yx}}(x,y) & \frac{\partial }{{\partial y}}\left( {\frac{{\partial z}}{{\partial y}}} \right) = \frac{{{\partial ^2}z}}{{\partial {y^2}}} = {f_{yy}}(x,y).
\end{align*}
\subsection{全微分}


\section{微分方程(该部分将会采用详细的讲义样式)}
\subsection{微分方程的基本概念}
\begin{definition}[微分方程的定义]
    一般地,凡表示\md{未知函数},\md{未知函数的导数}与\md{自变量之间的关系的方程},称为\textbf{微分方程}.
    其中未知函数的最高阶导数的阶数,称为\textbf{微分方程的阶}.一般地,$n$阶微分方程的形式是:\[
        F(
        x,y,y',\cdots,y^{(n)}
        )=0
    \]
\end{definition}
\begin{definition}[微分方程的解]
    设函数$y=\varphi(x)$在区间$I$上有$n$阶连续导数,如果在区间$I$上有:\[
        F[x,\varphi(x),\varphi'(x),\cdots,\varphi^{(n)}(x)]\equiv 0,
        \]那么函数$y=\varphi(x)$称为\textbf{微分方程$  F(x,y,y',\cdots,y^{(n)})=0$在区间$I$上的解}
    特别地,如果微分方程的解\md{含有任意常数}\footnote{此处的任意常数必须是相互独立的,或者说他们线性无关.},且任意常数的个数与微分方程的阶数相同,这样的解称为\textbf{微分方程的通解}.

    通解中时常含有任意常数,所以它还不能完全确定地反映某一客观事物的规律性.所以为了完全确定地反映客观事物的规律性,必须确定这些常数的值.为此要根据问题的实际情况,提出确定这些常数的条件.例如设一阶微分方程中的未知函数为$y=\varphi (x)$,通常给出的条件为$x=x_0,y=y_0$,也记为$y|_{x=x_0}=y_0$.

    因此我们定义,在实际问题中所给定的能够确定这些常数的条件称为\textbf{初值条件}.由初值条件确定了常数的值进而可以得到微分方程的\textbf{特解}.
\end{definition}
\section{可分离变量的微分方程}
本节我们将讨论一阶微分方程$y'=f(x,y)$
\end{document}
